\chapter{Introduction}
\label{chap:introduction}

\section{Background}
\label{section:background}

A background chapter in a book serves the purpose of providing essential context, information, or history that is relevant
to the overall understanding of the book's subject matter. This chapter typically appears early in the book and aims to set the
stage for the reader by offering background details that are crucial for comprehending the main narrative.

\section{Problem Statement}
\label{section:problem-statement}

A problem statement refers to a clear central issue,
challenge, or question that the project aims to address or explore. It is a
declaration that highlights the specific problem the author intends to examine,
discuss, or solve throughout the course of the project.

\section{Solution Overview}
\label{section:solution-overview}

A software solution overview provides a high-level and
concise description of a software product or system. It serves as an
introduction to the software, offering a glimpse into its key features,
functionalities, and the problems it aims to address. This overview is often
presented in documentation, marketing materials, or other communication
channels to give stakeholders, potential users, or decision-makers a quick
understanding of what the software does and why it is valuable.

\subsection{Features}
\label{subsection:features}

\begin{enumerate}[leftmargin=80pt]
    \item Feature Name: Short Description of Feature
    \item Feature Name: Short Description of Feature
\end{enumerate}

\section{Target User}
\label{section:target-user}

\begin{itemize}
    \item \textbf{Digital Content Creators \& Video Artists:} Professionals and amateurs (e.g., filmmakers, digital artists) concerned about AI scraping their work. Protractor ensures AI cannot accurately process their videos, preventing misuse.
    
    \item \textbf{Researchers in AI Security \& Adversarial Machine Learning:} AI ethics researchers and security analysts studying adversarial robustness. Since current AI poisoning methods mainly target images, Protractor provides a tool for testing adversarial perturbations in videos.
    
    \item \textbf{Legal and Copyright Protection Experts:} Media attorneys and digital rights organizations dealing with AI-generated deepfakes and content misuse. Protractor helps protect intellectual property from unauthorized AI training.
    
    \item \textbf{Industry Professionals in Media \& Entertainment:} Streaming platforms, game developers, and animation studios facing AI replication of artistic styles. Protractor’s poisoning techniques prevent AI from learning and mimicking their creative work.
\end{itemize}

\section{Benefit}
\label{section:benefit}

The Protractor system protects video content from unauthorized AI training by applying adversarial 
techniques that disrupt AI perception while remaining imperceptible to humans. It prevents AI from
 accurately processing videos, reducing the risk of deepfake generation and unauthorized replication. 
 This enhances intellectual property protection for creators and safeguards the creative industry 
 from AI-driven content theft. Additionally, Protractor supports AI security research, helping 
 experts study AI vulnerabilities in video poisoning. Its easy-to-use implementation allows content 
 creators and researchers to apply AI poisoning without requiring advanced technical expertise.

\section{Terminology}
\label{section:terminology}

Adversarial Attack: A method to manipulate AI models by adding small, unnoticeable changes that cause errors.

Breaking Temporal Consistency (BTC-UAP): A technique that disrupts AI’s ability to track motion across video frames.

Spatially Transformed Adversarial Attack (stAdv): Alters video structure to mislead AI while keeping it unchanged for humans.

Deepfake: AI-generated fake media that manipulates video content realistically.

AI Poisoning: Modifying data to mislead AI models and prevent accurate learning.

Perceptual Similarity Metrics (LPIPS \& SSIM): Measures that compare AI and human perception of video similarity.

Universal Adversarial Perturbation (UAP): Small changes in visuals that significantly affect AI recognition without altering human perception.