\chapter{AI Component}
\label{chap:ai-component}


\section{Task Requirements Analysis Using AI Canvas}
\label{section:Task-Requirements-Analysis-Using-AI-Canvas}
\subsection{AI Task Requirements}

\textbf{Objective:}  
To define what the AI poisoning system should achieve, how it will function, and the environment in which it will operate.

\subsubsection*{Requirements (REQ)}
\begin{itemize}
    \item REQ-1: The AI must generate adversarial examples from input images using the FGSM technique.
    \item REQ-2: The AI should ensure poisoned images cause significant misclassification in target neural network models.
    \item REQ-3: The poisoning should preserve visual integrity such that it remains perceptually similar to the original.
    \item REQ-4: The AI must support batch processing for datasets.
\end{itemize}

\subsubsection*{Specifications (SPEC)}
\begin{itemize}
    \item SPEC-1: The AI uses the Adversarial Robustness Toolbox (ART) to implement FGSM attacks.
    \item SPEC-2: The perturbation strength $\epsilon$ should be configurable by the user.
    \item SPEC-3: Poisoning will be applied directly to image tensors prior to model training.
\end{itemize}

\subsubsection*{Environment (ENV)}
\begin{itemize}
    \item ENV-1: The system runs in a Python environment (e.g., Jupyter or Colab) with TensorFlow or PyTorch backends.
    \item ENV-2: Input data consists of image files (.jpg, .png)
    \item ENV-3: Requires GPU acceleration for efficient batch poisoning.
    \item ENV-4: May be extended later for use in video frame sequences.
\end{itemize}

\subsection{AI Canvas Development}

\begin{table}[h]
\centering
\begin{tabular}{|p{4cm}|p{10cm}|}
\hline
\textbf{Canvas Element} & \textbf{Description} \\
\hline
\textbf{Input Data} & Raw image files. Input will be uploaded manually. \\
\hline
\textbf{Expected Output} & Poisoned images with imperceptible perturbations that cause target models miss understand. \\
\hline
\textbf{Tools/Frameworks} & Python, ART, TensorFlow or PyTorch, NumPy, Google Colab or local machine with GPU. \\
\hline
\textbf{Success Criteria} & 
\begin{itemize}
    \item Target models miss understand the feature.
    \item No noticeable visual difference in the poisoned training images (PSNR > 30dB).
\end{itemize} \\
\hline
\end{tabular}
\end{table}

\section{User Experience Design with AI}
\label{section:User-Experience-Design-with-AI}


\subsection{Interaction Style}
\textbf{Interaction Style: Annotate}

\begin{itemize}
    \item The AI modifies and tags poisoned images with visual indicators (e.g., warning labels).
    \item Users are informed when images are altered for adversarial purposes.
    \item This method provides transparency and keeps the user in control.
\end{itemize}

\subsection{User Feedback Mechanism}
Users are encouraged to provide feedback to help improve the AI poisoning system. The application offers a dedicated \textbf{Feedback} dropdown menu with the following options:
\begin{itemize}
    \item \textbf{Comments and Suggestions} – Share thoughts or ideas for improving the poisoning algorithm.
    \item \textbf{Bug Reports} – Report technical issues or incorrect behavior in the poisoning process.
    \item \textbf{Email Us} – Directly contact the development team for in-depth support.
\end{itemize}

\subsection{AI Contribution to System Intelligence}

This system integrates AI to perform intelligent poisoning of image data. 
This AI component significantly enhances the system’s capabilities beyond what 
is possible with non-AI approaches.

\subsubsection{Without AI Integration}
\begin{itemize}
    \item Manual image editing to distort data.
    \item Watermarking or fixed filters that are easily bypassed by neural networks.
\end{itemize}

\subsubsection{With AI Integration}
\begin{itemize}
    \item Automated poisoning of images based on gradient information.
    \item Adaptability to different neural network architectures.
    \item Preservation of visual quality for human viewers while deceiving models.
    \item Scalable batch processing of datasets with consistent adversarial effectiveness.
\end{itemize}
