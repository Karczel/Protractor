\newcounter{nuserstory}
\newcounter{nusecase}

\newcommand{\userstory}[4]{%
    \refstepcounter{nuserstory}
    \subsection{#1}
    \label{userstory:\thenuserstory}
    \hangindent=40pt
    \textbf{\textit{As a}} #2,\\
    \textbf{\textit{I want to}} #3,\\
    \textbf{\textit{so that}} #4.
}

\newcounter{nuserstory}
\newcounter{nusecase}

\newcommand{\userstory}[4]{%
    \refstepcounter{nuserstory}
    \subsection{#1}
    \label{userstory:\thenuserstory}
    \hangindent=40pt
    \textbf{\textit{As a}} #2,\\
    \textbf{\textit{I want to}} #3,\\
    \textbf{\textit{so that}} #4.
}
\newenvironment{usecase}[1]
{
    \refstepcounter{nusecase}%
    \subsection{Use Case \thenusecase: #1}%
    \label{usecase:\thenusecase}%
}{}

\chapter{Requirement Analysis}
\label{chap:requirement-analysis}

\section{Stakeholder Analysis}
\label{section:stakeholder-analysis}
<TIP: List your stakeholders for your project here/>

Stakeholders are individuals, groups, or entities that
have an interest, concern, or stake in a particular project, decision,
organization, or system. These are individuals or groups who can affect or be
affected by the outcomes of your project.

\section{User Stories}
\label{section:user-stories}
<TIP: Write user stories for each of your stakeholders here./>

User stories are a technique used in agile software
development to capture and describe functional requirements from an end
user's perspective. They are a way of expressing software features or
functionality in a simple, non-technical language that can be easily understood
by both developers and stakeholders.

\section{Use Case Diagram}
\label{section:use-case-diagram}

\begin{itemize}
    \item \textbf{\textit{User}}---A default group of users which has all the basic functionalities
\end{itemize}


\section{Use Case Model}
\label{section:use-case-model}

\begin{usecase}{Uploading a Video}
    \textbf{Actors:} Pong (Digital Content Creator), Protractor (System)

    \textbf{Description:} Pong wants to upload a video file to the system in preparation for poisoning.

    \textbf{Scenario:}
    \begin{enumerate}[leftmargin=80pt]
        \item Pong opens the Protractor web application.
        \item System displays the upload interface.
        \item Pong selects a supported video format file and clicks the upload button.
        \item System validates the file and stores it for processing.
    \end{enumerate}
    \textbf{Alternative Flow:} If the file type is unsupported, the system displays an error message.
\end{usecase}

\begin{usecase}{Setting Poisoning Parameters}
    \textbf{Actors:} Pong (Digital Content Creator), Protractor (System)

    \textbf{Description:} Pong customizes the poisoning settings before starting the video poisoning process.

    \textbf{Scenario:}
    \begin{enumerate}[leftmargin=80pt]
        \item Pong navigates to the settings panel after uploading a video.
        \item System displays available poisoning parameters.
        \item Pong selects the settings.
        \item System saves the selected configuration for use in the next step.
    \end{enumerate}
\end{usecase}

\begin{usecase}{Starting the Poisoning Process}
    \textbf{Actors:} Pong (Digital Content Creator), Protractor (System)

    \textbf{Description:} After configuring the parameters, Pong starts the video poisoning process.

    \textbf{Scenario:}
    \begin{enumerate}[leftmargin=80pt]
        \item Pong clicks the "Start Poisoning" button.
        \item System begins poisoning the video.
        \item System displays a progress indicator or loading bar.
    \end{enumerate}
\end{usecase}

\begin{usecase}{Viewing the Poisoning Progress}
    \textbf{Actors:} Pong (Digital Content Creator), Protractor (System)

    \textbf{Description:} Pong wants to track the progress of the poisoning process.

    \textbf{Scenario:}
    \begin{enumerate}[leftmargin=80pt]
        \item System updates a progress are processed.
        \item Pong monitors the process status.
        \item Once complete, the system notifies Pong that the video is ready.
    \end{enumerate}
\end{usecase}

\begin{usecase}{Downloading the Poisoned Video}
    \textbf{Actors:} Pong (Digital Content Creator), Protractor (System)

    \textbf{Description:} Pong downloads the final poisoned video file to their device.

    \textbf{Scenario:}
    \begin{enumerate}[leftmargin=80pt]
        \item The system displays a "Download" button once poisoning is complete.
        \item Pong clicks the button to download the poisoned video.
        \item System sends the processed file to Pong’s device.
    \end{enumerate}
\end{usecase}


\section{User Interface Design}
\label{section:user-interface-design}
<TIP: Put the initial design of your application here. You can
showcase a detailed design of a specific page or a sitemap of your application.
See an example below./>

\begin{figure}[h]
    \centering
    \includegraphics[width=0.8\textwidth]{examples/user-interface-design.png}
    \caption{User Interface Design}
\end{figure}